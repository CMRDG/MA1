%***********************************************************************
%* File            :    MAIN.tex
%*
%* HAUPTDOKUMENT
%*
%* Dieses Dokument bindet alle untergeordneten Dokumente
%* ein. Es wird beim Aufruf von pdflatex als Quelle angegeben.
%*
%* Autoren         :    Daniel Hering, Tobias Wartzek 
%*                      
%* Datum           :    05.12.2009
%***********************************************************************



\documentclass[12pt,a4paper,twoside,openright,index=totoc,listof=totoc,DIV15,BCOR=8.25mm,headinclude,footinclude=false,headsepline,notitlepage,numbers=noenddot,bibliography=totoc,headings=normal,parskip=full]{scrbook}

%************************************************************************
% Das Package muss leider hier eingebunden werden da sonst in den 
% Befehlen f�r hypersetup keinerlei Umlaute zur Verf�gung stehen
%************************************************************************

\usepackage[latin1]{inputenc}            % erlaubt direkte Eingabe Deutscher
                                         % Sonderzeichen in .tex Dateien
                                         %(latin1=iso8859-1 fuer Unix Editoren 
                                         % oder Windows Editoren die auch latin1
                                         % beherrschen)
                                         
%************************************************************************
% Dateiinfos eintragen (u.a. von hypersetup und f�r die Titelseite verwendet)
%************************************************************************
\usepackage{pdfpages}
% Titel der Arbeit 
\newcommand*{\doclongtitle}{Richtlinien f�r Abschlussarbeiten}

% Art der Arbeit (Diplomarbeit, Studienarbeit, Bachelorarbeit, etc.)
\newcommand*{\docsubject}{Masterarbeit}

% Name des Autors
\newcommand*{\docauthor}{Marian Walter}

% E-Mail Adresse des Autors
\newcommand*{\docauthoremail}{walter@hia.rwth-aachen.de}

% Schl�sselworte der Arbeit 
\newcommand*{\dockeywords}{MedIT Latex Studienarbeit Diplomarbeit Vorlage}

% Name des Betreuers
\newcommand*{\docsupervisor}{Dipl.-Ing. Mustermann}

\newcommand*{\docauthorfax}{+49 241 80-6-23211}
\newcommand*{\docauthorphone}{+49 241 80-23211}

% Bild auf der Titelseite
\newcommand*{\doctitlepic}{images/beispiel1}

% folgende Befehle nicht �ndern!!
\newcommand*{\dochomepage}{www.medit.hia.rwth-aachen.de}
\newcommand*{\docshorttitle}{\doclongtitle}

%**********************************************
%* Pakete (HEADER)                           *
%**********************************************

\include{HEADER}


%*************************************************************
%*************************************************************
%*																													**
%* 					Beginn des Dokuments 						  							**
%*																													**
%*************************************************************
%*************************************************************

\begin{document}
\begin{sloppypar}                       % sch�ner Blocksatz trotz langer Worte



%*************************************************************
%																														 *
%										 ENDE ALLER DEFINITIONEN 								 *
%																														 *
%*************************************************************


\frontmatter    												%r�mische Nummerierung aktivieren
\include{titelseite}
%\include{titelseite_alt}
\cleardoubleemptypage


%**************
%* Danksagung *
%**************

\chapter{Danksagung}
\thispagestyle{empty}
<!-- start slipsum code -->
Your bones don't break, mine do. That's clear. Your cells react to bacteria and viruses differently than mine. You don't get sick, I do. That's also clear. But for some reason, you and I react the exact same way to water. We swallow it too fast, we choke. We get some in our lungs, we drown. However unreal it may seem, we are connected, you and I. We're on the same curve, just on opposite ends.

Do you see any Teletubbies in here? Do you see a slender plastic tag clipped to my shirt with my name printed on it? Do you see a little Asian child with a blank expression on his face sitting outside on a mechanical helicopter that shakes when you put quarters in it? No? Well, that's what you see at a toy store. And you must think you're in a toy store, because you're here shopping for an infant named Jeb.

You see? It's curious. Ted did figure it out - time travel. And when we get back, we gonna tell everyone. How it's possible, how it's done, what the dangers are. But then why fifty years in the future when the spacecraft encounters a black hole does the computer call it an 'unknown entry event'? Why don't they know? If they don't know, that means we never told anyone. And if we never told anyone it means we never made it back. Hence we die down here. Just as a matter of deductive logic.
<!-- end slipsum code -->
\cleardoubleemptypage

%************************************************
%* Erkl�rung (hier muss nichts ge�ndert werden) *
%************************************************

\include{erklaerung}
\cleardoubleemptypage

%************
%* Abstract *
%************

\chapter{Zusammenfassung}

<!-- start slipsum code -->
Now that we know who you are, I know who I am. I'm not a mistake! It all makes sense! In a comic, you know how you can tell who the arch-villain's going to be? He's the exact opposite of the hero. And most times they're friends, like you and me! I should've known way back when... You know why, David? Because of the kids. They called me Mr Glass.

You see? It's curious. Ted did figure it out - time travel. And when we get back, we gonna tell everyone. How it's possible, how it's done, what the dangers are. But then why fifty years in the future when the spacecraft encounters a black hole does the computer call it an 'unknown entry event'? Why don't they know? If they don't know, that means we never told anyone. And if we never told anyone it means we never made it back. Hence we die down here. Just as a matter of deductive logic.

Now that we know who you are, I know who I am. I'm not a mistake! It all makes sense! In a comic, you know how you can tell who the arch-villain's going to be? He's the exact opposite of the hero. And most times they're friends, like you and me! I should've known way back when... You know why, David? Because of the kids. They called me Mr Glass.
<!-- end slipsum code --> 


\cleardoubleemptypage
\thispagestyle{empty}
\cleardoubleemptypage

%**********************
%* Inhaltsverzeichnis *
%**********************


\addcontentsline{toc}{chapter}{Inhaltsverzeichnis} % Eintrag im Inhaltsverzeichnis erstellen
\tableofcontents                        % Inhaltsverzeichnis anlegen
\cleardoubleemptypage



%*********************
%* Symbolverzeichnis *
%********************

\include{symbolverzeichnis}
\cleardoubleemptypage

\mainmatter							% Arabische Nummerierung, Beginn des Hauptteils

%**************
%* Einleitung *
%**************

\include{einleitung}

%*************
%* Hauptteil *
%*************
 
\include{richtlinien}

%**********
%* Anhang *
%**********

\cleardoublepage
\appendix
\include{anhang}


%**********************************************************
% Literaturverzeichnis (hier muss nichts ge�ndert werden) *
%**********************************************************

\bibliographystyle{alphadin}    %\bibliographystyle{} plain dinat
\bibliography{literatur}%


\end{sloppypar}
\end{document}
